%% ==========================================================================
%% UFRN - CCET - DIMAp
%%
%% Author.....: Selan R. dos Santos
%% Started on.: 28/08/2007
%%
%% Avaliador de Expressões Simples = AVES 
%% Compilador de Expressões SIMplificado - CESIM
%% Laboratório de Algoritmos e Estruturas de Dados - DIM0327
%% Tema: Uso de pilhas, manipulação de arquivos, ponteiros.
%% ==========================================================================


% ****************************************************************************
% Specifies the document class.
% -----------------------------
\documentclass[11pt,a4paper]{article}

% Specifies the packages.
\usepackage[portuges,brazil]{babel}
\usepackage[OT1]{fontenc}
%\usepackage[T1]{fontenc}
\usepackage{lmodern}
%\usepackage{ae}
\usepackage[ansinew]{inputenc}
%\usepackage{colortbl}
\usepackage{alltt}
\usepackage{fancyhdr}
%\usepackage{pst-node}
\usepackage{amssymb,amsmath}
\usepackage{array,booktabs}
%\usepackage{float}
\usepackage[margin=8pt,font={footnotesize,sf},labelfont=bf]{caption}
\usepackage[normalem]{ulem}
\usepackage{listings}
\usepackage{paralist} % itens dentro de parágrafo
%\usepackage{wrapfig}
%\usepackage{multirow}
\usepackage{mdwlist}
%\usepackage{algorithm2e}
%\usepackage{auto-pst-pdf}
\usepackage{float}
\usepackage[margin=10pt,font={footnotesize,sf},labelfont=bf]{caption}
\usepackage{listings}
%****************************************************************************
%\usepackage{pdfcolmk} % to avoid glitches caused by color in pdf
%\usepackage{graphicx}
%%\usepackage[usenames,dvipsnames]{color}
\usepackage{xcolor}
\usepackage{url}
\usepackage{booktabs}
\usepackage[linesnumbered]{algorithm2e}


 \definecolor{myaqua}{rgb}{0.0,0.5,0.55}
 \definecolor{lightaqua}{rgb}{0.75,0.95,0.95}

 \usepackage[colorlinks = true,
            linkcolor = myaqua,
            urlcolor  = blue,
            citecolor = myaqua]{hyperref}

\renewcommand{\familydefault}{\sfdefault}
\renewcommand{\baselinestretch}{1.2}

% DEFINIÇÕES ESPECIAIS

\setlength{\algomargin}{1.5em}

\SetKw{De}{de}
\SetKw{Passo}{com passo}
\SetKw{Inteiro}{inteiro}
\SetKw{Real}{real}
\SetKw{Caractere}{caractere}
\SetKw{Booleano}{booleano}
\SetKw{Texto}{texto}
\SetKw{Neutro}{neutro}
\SetKw{Nulo}{nulo}
\SetKw{End}{end}
\SetKw{Cte}{cte}
\SetKw{Var}{var}
\SetKw{Nao}{não}
\SetKw{E}{e}
\SetKw{OU}{ou}
\SetKw{Tipo}{tipo}
\SetKw{Referencia}{ref}
\SetKw{Ref}{ref}
\SetKw{Tamanho}{tam}
\SetKw{Arranjo}{arranjo de}
\SetKw{ArranjoParam}{arranjo}
\SetKw{Funcao}{função}
\SetKw{Procedimento}{procedimento}
\SetKw{Escreva}{escreva}
\SetKw{Leia}{leia}
\SetKw{Erro}{erro}
\SetKw{Membro}{membro}
\SetKw{Tupla}{tupla}
\SetKw{Aloque}{aloque}
\SetKw{Libere}{libere}
\SetKwBlock{TAD}{tad}{fim}
\SetKwBlock{Estrutura}{estrutura}{fim}
\SetKwBlock{Programa}{programa}{fim}
\SetKwBlock{DeclFunc}{função}{fim}
\SetKwBlock{DeclProc}{procedimento}{fim}
\SetKwBlock{Construtor}{construtor}{fim}
\SetKwBlock{Outro}{caso contrário}{fim}
%\SetKwSwitch{Selec}{Caso}{Outro}{caso}{seja}{igual a}{senão}{fim}
%\SetKwSwitch{Switch}{Case}{Other}{switch}{do}{case}{otherwise}{endsw}
\SetKwData{Falso}{falso}
\SetKwData{Verdadeiro}{verdadeiro}
\SetKwData{EstAluno}{Aluno}
\SetKwFunction{TADConstrutor}{\textnormal{\_\_}construtor\textnormal{\_\_}}
\SetKwComment{Comentario}{\textcolor{DarkGreen}{\#$\,$}}{}

% Various parameters #1
%\setlength{\textheight}{220mm}
%\setlength{\topmargin}{-3.4mm}
%\setlength{\textwidth}{160mm}
%\setlength{\oddsidemargin}{0.5mm}
%\setlength{\evensidemargin}{0.5mm}

% Various parameters #2
\setlength{\textheight}{229mm}
\setlength{\topmargin}{-5.4mm}
\setlength{\textwidth}{150mm}
\setlength{\oddsidemargin}{0mm}
\setlength{\evensidemargin}{0mm}

\definecolor{pink}{rgb}{1,0.5,0.5} % color values Red, Green, Blue
\definecolor{DarkGreen}{rgb}{0.0,0.5,0.0} % color values Red, Green, Blue
\definecolor{DarkBlue}{rgb}{0.0,0.0,0.3}
\definecolor{lightgrey}{rgb}{0.90,0.90,0.90}

%% Create a new float style, ideal for code fragments
\floatstyle{ruled}
\newfloat{program}{thp}{lop}
\floatname{program}{Algoritmo}


\floatstyle{ruled}
\newfloat{prog}{thp}{lop}
\floatname{prog}{Programa}

\lstset{
language=C++,
%basicstyle=\footnotesize\ttfamily,
basicstyle=\scriptsize\ttfamily,
%keywordstyle=\footnotesize\bfseries\sffamily,
keywordstyle=\scriptsize\bfseries\color{blue}\sffamily,
stringstyle=\scriptsize\color{red}\ttfamily,
commentstyle=\scriptsize\color{DarkGreen}\ttfamily,
morecomment=[l][\color{magenta}]{\#},
showstringspaces=false,
numbers=left,
numberstyle=\footnotesize,
stepnumber=1,
numbersep=5pt,
%backgroundcolor=\color{blue!05},
backgroundcolor=\color{gray!35},
showspaces=false,
showtabs=false,
escapeinside={(*}{*)}
}

% ******************************************************************************
% The preamble begins here.
% -------------------------
%\title{Empty}
%\author{Selan R. dos Santos}      % Declares the author's name.
%\date{05 de Agosto, 2005}      % Deleting this command produces today's date.

% ******************************************************************************
% The document starts here.
% -------------------------

\begin{document}
\pagenumbering{arabic}

\pagestyle{fancy}%
\renewcommand{\headrulewidth}{0.4pt}
\renewcommand{\footrulewidth}{0.4pt}
\lhead{\scriptsize{IMD0029 -- Estruturas de Dados Básicas I}}\rhead{\scriptsize Projeto ELIS, 2015.1}%
\cfoot{\scriptsize Página \thepage}%

% This affects itemize
%\renewcommand{\labelitemi}{$\star$}



\newcommand{\code}[1]{\texttt{\colorbox{lightgrey}{#1}}}



%% This is just to avoid the header on th 1st page.
\thispagestyle{empty}


% =============================================================================
% Cabeçalho Inicial.
% -----------------------------------------------------------------------------

\begin{center}
\Large\sc
Universidade Federal do Rio Grande do Norte\\
\large\sc
%Centro de Ciências Exatas e da Terra\\
Instituto Metrópole Digital \\*[0.3cm]

\rm
\normalsize
Estruturas de Dados Básicas I $\bullet$ IMD0029\\
  $\lhd$ Projeto de Programação $\rhd$\\[0.2cm]
  {\it Programa \textbf{ELIS} ({\it \textbf{E}ditor de textos orientado a \textbf{LI}nha\textbf{S}})}\\[0.2cm]% Declares the document's title.
%22 de julho de 2014\\*[1.0cm]
\today\\*[1.0cm]
\end{center}


\section*{Apresentação}
\label{sec:objetivos}

O objetivo deste projeto de programação é utilizar as estruturas de dados do
tipo \emph{lista encadeada} para a resolução de problemas práticos.
Um dos pré-requisitos deste exercício é que estas estruturas de dados já estejam
implementadas na forma de classes genéricas em C++.
%
A aplicação do TAD \emph{lista encadeada} será na construção de um editor de texto
orientado a linhas.

\tableofcontents

\pagebreak
\section{Introdução}
\label{subsec:parteI:descricao}

Um dos primeiros tipos de programas para editar textos eram orientados a linhas,
ou seja, editava-se uma linha por vez (consultar verbete Wikipédia sobre
\href{http://en.wikipedia.org/wiki/Ed_(text_editor)}{Ed}).
Tais editores, denominados de \emph{editores orientado a linhas}, possuíam uma interface
com usuário bem restrita mas, por outro lado, eram bem versáteis e
adequados aos terminais textuais típicos da época do surgimento do computador.

Posteriormente, estes editores evoluíram para os chamados \emph{editores de página cheia}.
Mesmo assim, a influência dos primeiros editores de linhas foi tamanha que ainda
hoje um dos editores preferidos pela comunidade de programadores, o \texttt{vi} ou \texttt{vim}, ainda
utiliza uma sintaxe de comandos bem similar aos antigos editores de linhas.

\section{Tarefa}
\label{subsec:tarefa_parteI}

Sua tarefa consiste em desenvolver um editor de linhas, denominado
\texttt{elis} (\uline{E}ditor de \uline{LI}nha \uline{S}imples),
que armazena o texto em uma lista encadeada, cada linha em um nó separado de uma
lista encadeada de cadeia de caracteres (\emph{strings}).

O programa \texttt{elis} opera em três modos exclusivos e distintos:
\textsl{normal}, \textsl{edição} ou \textsl{comando}.
No modo de \textsl{edição} deve ser exibido o \emph{prompt} `$\sim$ INSERT $\sim$`, para indicar qual é o estado atual. Neste modo é possível entrar com texto para compor uma linha.
Pressionando-se \texttt{<ENTER>} a linha é finalizada e armazenada na lista
encadeada; a seguir uma nova linha é automaticamente disponibilizada para edição.

A Figura~\ref{fig:1} demonstra a utilização do programa para a criação
de um arquivo fornecido como parâmetro (no caso, ``readme.txt'').
Neste exemplo o programa está atuando apenas no modo \textsl{edição}.

% ==============================================================================
% Figura 2
% ------------------------------------------------------------------------------
\begin{figure}[!ht]
% ------------------------------------------------------------------------------

\begin{lstlisting}[numbers=none]
(*\$*) ./elis readme.txt
1> Esta é a primeira linha(*$\hookleftarrow$*)
2> enquanto que esta é a segunda linha(*$\hookleftarrow$*)
3> mais uma linha(*$\hookleftarrow$*)
4> (*$\hookleftarrow$*)
5> última linha editada(*$\hookleftarrow$*)
\end{lstlisting}
% -----------------------------------------------------------------------------
   \caption  {Exemplo de uso do \texttt{elis} no modo \textsl{edição}.
       O símbolo `$\hookleftarrow$' representa o pressionamento do \texttt{<ENTER>}.}
   \label{fig:1}
\end{figure}
% ==============================================================================
%\curvearrowleft

	Para sair do modo \textsl{edição} e entrar no modo \textsl{normal} o usuário deve pressionar \texttt{<ESC>}, fazendo com que o programa exiba o \emph{prompt} de comando `$\sim$ NORMAL $\sim$'.
Uma vez no modo \textsl{normal} o usuário pode utilizar \texttt{<j-k>} para descer ou subir de linha, respectivamente. Para sair do modo \textsl{normal} e entrar no modo \textsl{comando} o usuário deve pressionar \texttt{<:>}, mostrando o \emph{prompt} de comando `$\sim$ COMMAND INSERT $\sim$`. Enfim, enquanto neste modo o cliente pode efetuar qualquer um dos comandos descritos
na Tabela~\ref{tab:cmds}.


\begin{table}[htb!]
\begin{center}
\footnotesize
\begin{tabular}{p{1.9cm}  p{12cm}} \toprule
\textbf{\sc Comando}                       & \textbf{\sc Descrição} \\ \midrule
\rule[-2mm]{0mm}{6mm}\texttt{W [<name>]}   & Salva todas as linhas do texto em um arquivo ascii \texttt{name}.
                                             O comando sem o fornecimento de um nome simplesmente grava o texto no arquivo atual.
                                             Se o nome do arquivo atual ainda não foi fornecido o programa deve solicitar um nome ao usuário.\\
\rule[-2mm]{0mm}{6mm}\texttt{E <name>}     & Lê para a memória todas as linhas de texto do arquivo ascii \texttt{name}.
                                             Se o arquivo indicado não existir um novo arquivo vazio \texttt{name} deve ser criado.\\
\rule[-2mm]{0mm}{6mm}\texttt{I [n]}        & Entra no modo de \textsl{edição}, permitindo a inserção de texto \emph{antes} da linha \texttt{n}.
                                             Se \texttt{n} não é fornecido, o texto é inserido \emph{antes} da linha atual.\\
\rule[-2mm]{0mm}{6mm}\texttt{A [n]}        & Entra no modo de \textsl{edição}, permitindo a inserção de texto \emph{depois} da linha \texttt{n}.
                                             Se \texttt{n} não é fornecido, o texto é inserido \emph{depois} da linha atual.\\
\rule[-2mm]{0mm}{6mm}\texttt{M [n]}        & Torna \texttt{n} a linha atual.
                                             Se \texttt{n} não é fornecido então a última linha do texto passa a ser a atual.\\
\rule[-2mm]{0mm}{6mm}\texttt{D [n [m]]}    & Remove linhas \texttt{n} até \texttt{m}.
                                             Se apenas \texttt{n} é fornecido, remove-se a linha \texttt{n}.
                                             Se nenhum número é fornecido, remove-se a linha atual.\\
\rule[-2mm]{0mm}{6mm}\texttt{H}            & Exibe um texto de ajuda, explicando de forma resumida quais são os comandos do programa.\\
\rule[-2mm]{0mm}{6mm}\texttt{Q}            & Encerra o programa. Se o texto atual não tiver sido salvo, o programa deve exibir uma mensagem indicando o fato e confirmar a operação.\\

\bottomrule
\end{tabular}
\end{center}
\normalsize
\caption{Lista de comandos do programa \texttt{elis}. O símbolo `[{\ }]' indica os argumentos opcionais.}

\label{tab:cmds}
\end{table}

A Figura~\ref{fig:2} apresenta um exemplo de interação com o programa \texttt{elis},
no qual algumas operações foram executadas sobre o texto criado na Figura~\ref{fig:1}.
Os comandos podem ser fornecidos através de letras maiúsculas ou minúsculas.

% ==============================================================================
% Figura 2
% ------------------------------------------------------------------------------
\begin{figure}[!ht]
% ------------------------------------------------------------------------------

\begin{lstlisting}[numbers=none]
1> Esta é a primeira linha
2> enquanto que esta é a segunda linha
3> mais uma linha
4>
5*> última linha editada(*$\looparrowright$*)
: I 3
1> Esta é a primeira linha
2> enquanto que esta é a segunda linha
3> nova linha inserida(*$\hookleftarrow$*)
4*> mais uma linha inserida com o comando 'I'!(*$\looparrowright$*)
5> mais uma linha
6>
7> última linha editada(*$\looparrowright$*)
: D 6
1> Esta é a primeira linha
2> enquanto que esta é a segunda linha
3> nova linha inserida
4*> mais uma linha inserida com o comando 'I'!(*$\looparrowright$*)
5> mais uma linha
6> última linha editada
: A 6
1> Esta é a primeira linha
2> enquanto que esta é a segunda linha
3> nova linha inserida
4> mais uma linha inserida com o comando 'I'!
5> mais uma linha
6> última linha editada
7> inserindo linha depois(*$\hookleftarrow$*)
8*>(*$\looparrowright$*)
: W
: Q
\end{lstlisting}
% -----------------------------------------------------------------------------
   \caption  {Exemplo de uso do \texttt{elis} no modo \textsl{comando}.
       O símbolo `$\hookleftarrow$' representa o pressionamento do \texttt{<ENTER>} e
       o símbolo `$\looparrowright$' representa o pressionamento do \texttt{<ESC>}.}
   \label{fig:2}
\end{figure}
% ==============================================================================
%\curvearrowleft

Note que a numeração que aparece na primeira coluna, por exemplo  ``\texttt{3>}'',
serve apenas para identificar a ordem da linha no texto e, desta forma, não
faz parte do texto em si.
Portanto, esta numeração \textbf{não} deve aparecer no arquivo ascii gravado pelo programa
\texttt{elis}.

De acordo com a descrição dos comandos apresentados na Tabela~\ref{tab:cmds}, um
importante conceito é o de \textbf{linha atual}.
A linha atual consiste em uma linha ``selecionada'' sobre o qual as operação serão
realizadas, caso nenhuma outra linha seja indicada.
O programa \texttt{elis} sempre possui uma linha atual ativada, normalmente a ultima
linha editada.
O programa deve sinalizar para o usuário qual a linha atual através de um `*' logo
após a indicação do número da linha, como em ``\texttt{7*>}'' no final da Figura~\ref{fig:2}.


\section{Desafios de Implementação}
\label{subsec:parteI:desafios}

% DANIEL MEXEU
% APAGAR TUDO DESDE LINHA EM DIANTE ATÉ "DANIEL PAROU".
\subsection{Pré-Requisitos}

Como forma de facilitar a visualização em terminal do projeto, foi adotado pelos monitores uma estratégia utilizando a biblioteca \texttt{ncurses} (para saber mais sobre a biblioteca \href{http://bit.do/eAJR6}{ncurses}) do \texttt{C++}.

% Eles que pesquisem como instalar.

%TODO LINK PARA TEASER

Esta estratégia consiste na disponibilização para vocês de uma pequena \texttt{API}, que deverá facilitar o trabalho quanto à visualização gráfica do projeto. Em outras palavras, a tarefa de vocês focará integralmente na criação das \texttt{TADs} adequadas para a programação \texttt{back-end} do projeto. Recomenda-se olhem a \texttt{API} para saberem como utilizá-la. Este é um \href{}{\textit{teaser}} mostrando o \texttt{elis} em funcionamento com \emph{ncurses}. 

No entanto, notem que ainda lhes caberá a utilização desta \texttt{API}. Será enviado à vocês um arquivo contendo as funcões disponibilizadas. \textbf{Importante lembrá-los da chamada das funções dadas, pois vocês devem saber quando executá-las, conforme novas informações forem sendo inseridas.}
% Disponibilizar funções

\subsection{Implementação}

% DANIEL PAROU AQUI
Um dos elementos de desafio deste trabalho é escolher a estrutura de dados mais
apropriada.
Para melhor identificar as estruturas necessárias, tente antecipar todas as
operações que serão necessárias.
A partir desta informação, decida qual estrutura de dados utilizar.
Leve em consideração fatores como a eficiência geral do programa (complexidade temporal) e o
consumo de memória (complexidade espacial).

O próximo desafio consiste em desenvolver sua própria rotina de leitura de caracteres
para formar uma cadeia.
Isto deve ser feito de forma a permitir a captura de teclas especiais, como \texttt{<ENTER>}
e \texttt{<ESC>}.
%O Apêndice~\ref{apendice:1} fornece um programa como ponto de partida.
%Este programa realiza a leitura de uma cadeia de caracteres,
%um caractere por vez.

Para facilitar o desenvolvimento do projeto, é muito importante compreender o
\href{http://en.wikipedia.org/wiki/State_diagram}{\emph{diagrama de estados}} que
o \texttt{elis} pode assumir em execução.
Veja, por exemplo, este \href{https://darkpan.com/files/vim.svg}{diagrama de estados do \texttt{vim}}.
Portanto, recomenda-se a elaboração de um diagrama de estados para o \texttt{elis} antes de iniciar
a implementação do projeto.

Outro desafio consiste em prever e tratar de forma apropriada o maior número possível
de erros de interação usuário-programa.
Por exemplo, o que acontece se o usuário pressionar ``\texttt{:A 10}'' em um arquivo
que contém apenas 2 linhas?
Ou então se o usuário fornecer o comando ``\texttt{:D 10 5}'', devemos indicar
um erro ou deduzir que o programa vai apagar da linha 5 até a 10?
Pensar em uma boa interface e um tratamento de erros robusto é fundamental para
o desenvolvimento de \emph{software} de qualidade.

Pontos extras estão disponíveis apenas para os trabalhos \textbf{completos}.
Isso quer dizer que os projetos que implementaram todas as funcionalidades
descritas neste documento podem ganhar mais pontos se ampliarem a funcionalidade
do \texttt{elis}.
Isto pode ser feito de várias maneiras, como por exemplo 
acrescentando comandos para procurar (\textit{find} ou \texttt{F}) palavras
ou fragmentos de palavras, procurar e substituir, copiar linhas, prover
suporte para desfazer (\textit{undo} ou \texttt{U}), etc.

\section{Avaliação do Programa}
\label{sec:avaliacao}

Para a implementação deste projeto \textbf{é obrigatório} a utilização das
classes correspondente a estruturas de dados que foram apresentadas em
sala de aula.
Não serão aceitas soluções que utilizem as estruturas de dados da biblioteca
externas, como STL (e.g.\ \code{list}, \code{vector}, etc.)
ou \texttt{boost}, por exemplo.

O programa completo deverá ser entregue sem erros de compilação, testado e totalmente
documentado.
O programa \texttt{elis} será avaliado sob os seguintes critérios:-
\begin{itemize*}
    \item Comando W funciona corretamente ($10 \%$)
    \item Comando H funciona corretamente ($5 \%$)
    \item Comando E funciona corretamente ($10 \%$)
    \item Comando I funciona corretamente ($15 \%$)
    \item Comando A funciona corretamente ($15 \%$)
    \item Comando M funciona corretamente ($5 \%$)
    \item Comando D funciona corretamente ($20 \%$)
    \item Trata de maneira compreensiva possíveis erros de entrada ($10 \%$)
    \item Funcionamento geral correto, como por exemplo ser capaz
        de alterar entre os estados de edição e de comandos.($10 \%$)
\end{itemize*}

A pontuação acima não é definitiva e imutável.
Ela serve apenas como um guia de como o trabalho será avaliado em linhas gerais.
É possível a realização de ajustes nas pontuações indicadas visando adequar a pontuação
ao nível de dificuldade dos itens solicitados.

Os itens abaixo correspondem à descontos, ou seja, pontos que podem ser retirados
da pontuação total obtida com os itens anteriores:-
\begin{itemize*}
    \item[$\circ$] Presença de erros de compilação e/ou execução (até $-20 \%$)
    \item[$\circ$] Falta de documentação do programa com Doxygen (até $-10 \%$)
    \item[$\circ$] Vazamento de memória identificado com o valgrind (até $-10 \%$)
    \item[$\circ$] Falta ou incompletude do arquivo \texttt{README.md} (até $-10 \%$)
\end{itemize*}

Você deve escrever um arquivo \texttt{README.md}
(formato \href{http://daringfireball.net/projects/markdown/syntax}{Markdown}) com, pelo menos, informações sobre
como o programa foi implementado, i.e.\ quais estruturas de dados foram utilizadas,
explicações sobre o funcionamento de cada um de seus comandos (com exemplos),
indicação dos componentes da equipe desenvolvedora (com email),
instruções para compilação e instalação.
Fique à vontade para incluir no arquivo \texttt{README.md} qualquer outra informação
que a equipe julgar relevante.

\subsection*{Boas práticas de programação}
Recomenda-se fortemente o uso das seguintes ferramentas:-
\begin{itemize*}
    \item Doxygen: para a documentação de código e das classes;
    \item Git: para o controle de versões e desenvolvimento colaborativo;
    \item Valgrind: para verificação de vazamento de memória;
    \item gdb: para depuração do código; e
    \item Makefile: para gerenciar o processo de compilação do projeto.
\end{itemize*}

Recomenda-se também que sejam realizados \href{http://en.wikipedia.org/wiki/Software_testing}{testes}
de utilização do programa em várias situações.
Procure organizar seu código em várias pastas, conforme vários exemplos apresentados em
sala de aula, com pastas como
\texttt{src} (arquivos \texttt{.cpp}),
\texttt{include} (arquivos \texttt{.h}),
\texttt{bin} (arquivos \texttt{.o} e executável) e
\texttt{data} (arquivos de entrada e saída de dados).

Uma forma de validar o seu programa é inserir diretivas de compilação condicional
para compilar o seu projeto ora usando suas classes (por exemplo, \code{Lista}), ora
usando classes equivalentes do STL (\code{vector}, \code{list}, etc.).
Esta estratégia permite isolar erros no programa \texttt{elis} de erros na implementação
das classes básicas.

\section{Autoria e Política de Colaboração}
%
O trabalho pode ser realizado \textbf{individualmente} ou em \textbf{duplas},
sendo que no último caso é importante, dentro do possível,
dividir as tarefas igualmente entre os componentes.

Qualquer equipe pode ser convocada para uma entrevista.
O objetivo da entrevista é duplo: confirmar a autoria do trabalho e
determinar a contribuição real de cada componente em relação ao trabalho.
Durante a entrevista os membros da equipe devem ser capazes de explicar,
com desenvoltura, qualquer trecho do trabalho, mesmo que o código tenha
sido desenvolvido pelo outro membro da equipe.
Portanto, é possível que, após a entrevista, ocorra redução da nota geral do trabalho ou
ajustes nas notas individuais, de maneira a refletir a verdadeira contribuição
de cada membro, conforme determinado na entrevista.

O trabalho em cooperação entre alunos da turma é estimulado.
É aceitável a discussão de ideias e estratégias.
Note, contudo, que esta interação \textbf{não} deve ser entendida como permissão
para utilização de código ou parte de código de outras equipes,
o que pode caracterizar a situação de plágio.
Em resumo, tenha o cuidado de escrever seus próprios programas.

Trabalhos plagiados receberão nota \textbf{zero} automaticamente,
independente de quem seja o verdadeiro autor dos trabalhos infratores.
Fazer uso de qualquer assistência sem reconhecer os créditos apropriados
é considerado \textbf{plagiarismo}.
Quando submeter seu trabalho, forneça a citação e reconhecimentos necessários.
Isso pode ser feito pontualmente nos comentários no início do código, ou,
de maneira mais abrangente, no arquivo texto \texttt{README.md}.
Além disso, no caso de receber assistência, certifique-se de que ela lhe
é dada de maneira genérica, ou seja, de forma que não envolva alguém
tendo que escrever código por você.


\section{Entrega}
\label{sec:entrega}

Você deve submeter um único arquivo com a compactação da pasta do seu projeto.
Se for o caso, forneça também o link Git para o seu projeto.
O arquivo compactado deve ser enviado \textbf{apenas} através da opção Tarefas
da turma Virtual do Sigaa, em data divulgada no sistema.
%\textbf{Note que qualquer envio de trabalho por email será desconsiderado!}


%----------------------------------------------------------------------

\center $\blacktriangleleft$  FIM $\blacktriangleright$

\end{document}               % End of document.
